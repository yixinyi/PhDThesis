\chapter{Localization}
Localization of an integral happens when the integral is exactly equal to its saddle-point approximation.
% The localization loci are then the saddle-points.
A trivial example is therefore the Gaussian integration formula, which the saddle-point method is based on:
\begin{equation}\label{GaussianIntegral}
 \int_{\mathbb{R}^n} d^n x \, e^{-\frac{1}{2} x^T A x} = \sqrt{\dfrac{(2\pi)^n}{\text{det}A}},
\end{equation}
where $A$ is a positive-definite $n\times n$ matrix.


In the case of path integrals in quantum field theories, which are of infinite dimension,
localization, if applicable, reduces them to finite dimensional integrals.
% This is achieved using functional generalizations of the Gaussian formula \eqref{GaussianIntegral},
% and the resulting integral is over the moduli space of the theory vacua.
This is an exact approach, in contrast to perturbative methods such as Feynman diagrams,
valid only in the weakly-coupled regime.
Unfortunately, very few path integrals are solvable, 
but there are very specific classes of theories where localization does apply.
These are supersymmetric theories and topological quantum field theories,
from which we distinguish two types of localization:
supersymmetric localization and equivariant localization.
% As a matter of language, and
% this effect is often said to be \emph{semiclassically-exact}, or \emph{1-loop exact}.

Many results of this thesis are based on the supersymmetric localization.
We will review its basic idea and illustrate it explicitly in an example with an ordinary integral. 
We refer the reader to \cite{Szabo:1996md} and \cite{Pestun:2016zxk} for reviews of the topic.


\section{Supersymmetric localization}

In the path integral formulation of quantum theories, 
physical observables are expectation values of operators.
Let us consider the path integral (in the Euclidean signature and set $\hbar = 1$):
\begin{equation}
 \braket{O} = \int D\phi O \, e^{-S[\phi]},
\end{equation}
where $O$ is an operator, built out of the fields of the theory, represented by $\phi$.

Assume we can deform the action $S$ in such a way that it does not affect the physical quantity, 
namely
\begin{equation}
 \braket{O}_t = \int D\phi e^{-S[\phi]-t Q V[\phi]}, 
\end{equation}
where 
% $QV\geq 0$, and 
$Q$ is a fermionic symmetry of the theory, and require
\begin{equation}
 \dfrac{d \braket{O}_t}{d t} = 0.
%  \overset{!}{=} 0.
\end{equation}
Integrating by parts and assuming vanishing boundary terms, 
we obtain the following necessary conditions:
\begin{equation}\label{loc:susyCondition}
 Q^2 = 0, \quad Q S = 0, \quad Q O = 0,
\end{equation}
and that the measure $D\phi$ is also invariant (the symmetry must not be anomalous).

If this deformation were possible, then the deformation parameter $t$ could be of any value. 
The most useful case is to take it to be infinitely large,
in order to use the saddle-point approximation.
Since by construction the integral is independent of $t$, the saddle-point approximation must be exact,
hence the path integral localizes to a set of loci.
These are determined by\footnote{ 
If the bosonic part is positive-definite.
The fermionic contribution of the localization action is subleading.}
\begin{equation}\label{loc:saddlePoint}
 (QV)_\text{Bosonic} = 0
\end{equation}
whose solutions---let us denote them by $\phi_*$---belong to the moduli space of vacua of the theory.
This means that some fields acquire a non-zero vacuum expectation value.
Accounting the fluctuations around the vacuum configuration,
the localized integral can be formally written as:
\begin{equation}
 \braket{O}=\int_\text{Moduli} D\phi_* O(\phi_*) e^{-S(\phi_*)} Z_\text{1-loop},
\end{equation}
where the integral over $\phi_*$ is analogous to the sum over all the saddle-points, 
and $Z_\text{1-loop}$  is the functional generalization of the Gaussian integral \eqref{GaussianIntegral} for the fluctuations around the saddle-points. 
It is thus a functional determinant, which is divergent in general.
In order to obtain a finite result, the theory is defined in a compact manifold, like a sphere, 
such that the spectrum of the operator is discrete, and supersymmetry guarantees the cancellation of divergences between bosons and fermions.
Despite conceptual simplicity, the main challenge of this technique is to find the localization action $QV$, since no general recipe exists. 



\section{A toy example}

Consider an ordinary integral
\begin{equation}
 I = \int_m^M dx \,p g'(x)\,e^{p g(x)}, 
\end{equation}
with $p$ being a constant.
It can be solved exactly as a total derivative:
\begin{equation}
 I = \int_m^M  dx \, \dfrac{d}{d x}\,e^{p g(x)} 
   = e^{p g(M)}-e^{p g(m)}.
\end{equation}


If we were to solve it using the saddle-point approximation for $p \rightarrow \infty$,
then the saddle-points must be the endpoints $x_* = \{m, M\}$.
Let us consider this case and expand around the saddle-points:
\begin{equation}
 g(x_*+\xi) = g(x_*)+\frac{1}{2} g''(x_*) \xi^2 +\mathcal{O}(\xi^3).
\end{equation}
Each saddle-point contributes to the integral as following:
\begin{eqnarray}
 I_m &=& e^{p g(m)} \int_0^\infty  d\xi \, p g''(m) \xi\, e^{p \frac{1}{2} g''(m) \xi^2 } 
      =- e^{p g(m)} \\
 I_M &=& e^{p g(M)} \int_{-\infty}^0 d\xi \, p g''(M) \xi\, e^{p\frac{1}{2} g''(M) \xi^2 } 
      = e^{p g(M)}   
\end{eqnarray}
valid for $\Re(p g'') < 0$.
The sum $I_m + I_M$ indeed gives the exact result.
% In conclusion, the integral $I$ is a trivial example of localization.


Now, let us see how we can use supersymmetric localization to solve it. 
The integral $I$ can be rewritten in the supersymmetric form:
\begin{equation}
 I = \int_m^M \, dx \int da \int db \, e^{p(g(x)- a b g'(x))},
\end{equation}
where $a$ and $b$ are Grassmann numbers (our fermions), 
which means they satisfy $a b = -b a$ and $a^2=b^2=0$, hence 
\begin{equation} \label{GrassmannIntegral}
 f(x) = \int da \int db \, e^{-a\, f(x) \,b }.
\end{equation}


The action $S=p(g(x)-a b g'(x))$ is invariant under the supersymmetry transformation:
\begin{equation}
 \delta_\epsilon x = -\epsilon a, 
 \quad
 \delta_\epsilon a = 0,
 \quad 
 \delta_\epsilon b = \epsilon,
\end{equation}
where $\epsilon$ is a Grassmann number.
The supersymmetric operator, defined by $\epsilon Q = \delta_\epsilon$,
is explicitly
\begin{equation}
 Q = -a \frac{\partial }{\partial x}+\frac{\partial }{\partial b}.
\end{equation}
We can check that, indeed, \eqref{loc:susyCondition} are fulfilled.


Let us deform the integral with a $Q$-exact term
\begin{equation}
 I(t)=\int_m^M \, dx \int da \int db \, e^{p(g(x)-a b g'(x))+ t Q V},
\end{equation}
where 
\begin{equation}
 V(x, a, b) = f(x) b
\end{equation}
that leads to a non-trivial localization action
\begin{equation}
 QV = f(x)-a b f'(x).
\end{equation}
This is actually the most general supersymmetric action we can write for this case.
Thus, $S$ is not just $Q$-closed (i.e. $QS=0$), but also $Q$-exact (i.e. $S=QV_s$).


We require the deformed integral to be independent of the deformation parameter $t$:
\begin{equation}
 I'(t)=0.
\end{equation}
An explicit and straightforward computation shows
\begin{eqnarray}
 I'(t) &=& \int_m^M \, dx \int da \int db \, e^{S + t Q V} Q V \\
       &=& \int_m^M \, dx \int da \int db \, Q (e^{S + t Q V} V)\\
       &=& \left. e^{p g(x)+ t f(x)} f(x) \right|^M_m
\end{eqnarray}
where we used the supersymmetry condition and nilpotency. 
In this case, we must additionally require vanishing boundary conditions
\begin{equation}
 f(m)=f(M)=0.
\end{equation}

Now we can take large $t$ to solve the integral using the saddle-point method.
The boundary points here must be either global maxima or global minima of $f(x)$,
for $t>0$ or $t<0$, in order for the saddle-point integral to be convergent.
In other words, for $t$ negative (positive), 
the bosonic part of the localization action is positive (negative) definite, 
and it vanishes at the saddle-points:
\begin{equation}
 (QV)_\text{Bosonic}=0.
\end{equation}

The saddle-point approximation is exact, in the same fashion as for the large $p$ case we studied before.
This is to be expected, 
since the deformed integral is still a total derivative when we integrate out the fermions:
\begin{equation}
 I(t) = \int_m^M  dx \, \dfrac{d}{d x}\,e^{p g(x) + t f(x)} 
      = e^{p g(M)+ t f(M)}-e^{p g(m)+ t f(m)}.
%       = e^{p g(M)}-e^{p g(m)}.      
\end{equation}

When $g(x)=\cos x$ and the integration region is extended to a sphere 
parametrized by the polar angle $ x\in [0,\pi]$ and the azimuthal angle $\varphi \in [0, 2\pi]$,
this becomes a particular example of the Duistermaat-Heckman integration formula,
which is the precursor of the localization of path integrals.

% Witten used this example to illustrate the equivariant localization in the appendix of \cite{Witten:1992xu}.

% \begin{equation}
%   \int_0^{2\pi} d\phi \, \int_0^{\pi} \, d\theta \sin\theta \, e^{t \cos\theta}
% = \dfrac{2\pi}{t}\left(e^{t} - e^{-t}\right).
% \end{equation}
% The localization loci are the north and south pole, namely $\theta_*=\{0, \pi\}$.


% The theorem was later proved to be a special case of a more general localization property of equivariant cohomology
% and Berline and Vergne derived a localization formula for general compact Riemannian manifolds.
% The first infinite-dimensional generalization of the theorem,
% in the setting of a supersymmetric path integral,
% was worked out by Atiyah and Witten.


