\chapter{Conclusions}



In this thesis, 
we explored a very special class of supersymmetric gauge theories whose path integrals are directly reducible to finite-dimensional integrals, 
i.e. 0-dimensional matrix models, by the means of the supersymmetric localization method.
In particular, we studied the vacuum partition function and the expectation values of BPS Wilson loop observables of $\mathcal{N}=2^*$ SYM on $S^4$.


Our main motivation was to rigorously test the gauge/string duality conjecture 
in a slightly more generalized setting than the well-known case of $\mathcal{N}=4$  SYM and type IIB strings on $AdS_5 \times S^5$, 
by introducing a mass scale. 
The total access to the strong coupling phase of $\mathcal{N}=2^*$ SYM gives us a golden opportunity to explicitly test it 
against its presumed dual theory of type IIB strings on the Pilch-Warner background.

Computations are highly non-trivial, 
on the gauge theory side, 
as well as on the string theory side. 
From our work, we can conclude the following, supplemented with possible future work:
\begin{itemize}
 \item Paper I and \cite{Zarembo:2014ooa} showed that the decompactifaction limit of $\mathcal{N}=2^*$ SYM on $S^4$ 
 commutes with the strong coupling limit. 
 This allows us to do holographic studies, 
 since the dual of $\mathcal{N}=2^*$ SYM on $\mathbb{R}^4$ is known. 
 For the sake of completeness, it would be interesting to extend the holographic study for the theory on $S^4$.
 
 \item 
%  There are infinitely-many phase transitions previously seen for finite critical couplings at the decompactification limit \cite{Russo:2013qaa}. 
 We used BPS Wilson loops in symmetric and antisymmetric representations of the gauge group to probe the strong-coupling phase of $\mathcal{N}=2^*$ SYM on $S^4$,
 especially in the decompactifaction limit, where infinite cusps appear.
 Paper II showed that these cusps induce phase transitions seen in the subleading order in strong coupling for these Wilson loops.
 In order to understand the nature of these phase transitions in the string theory side, 
 Paper III computed the D3-brane configuration in the Pilch-Warner background dual to the symmetric Wilson loop. 
 Only the leading result was derived, and it does not probe the full field theory result regime.
%  A different dual interpretation is needed to fully probe the field theory side solution though. 
%  Therefore, we have not been able to answer the question of the nature of the holographic description of the phase transitions. 
 From scaling arguments, we believe the D5-brane dual to the antisymmetric representation can be a promising case 
 for a full matching, and hopefully even at 1-loop level, where phase transitions happen. 
%  Unfortunately, due to technical difficulties, the solution has been elusive, 
% 
%  An attempt of a computer algebra program has been used to systematically solve the supersymmetric equations, 
%  and a new unpublished D7 brane solution has been found.
 
 
 \item The gauge/string duality for $\mathcal{N}=2^*$ case works with the fundamental Wilson line at least up to 1-loop quantum corrections, 
 shown numerically by Paper V. Given the simplicity of the field theory prediction, 
 an analytical result in the string theory side might be possible and desirable. 
 Moreover, the matching requires the existence of the controversial Fradkin-Tseytlin term in the string action. 
 It would be clarifying to prove its existence independently, for example by requiring the beta function of the Green-Schwarz action to be zero.
 
  \item It is known that Wilson loops in the symmetric representation and the multiply-wrapped representation differ 
 by exponentially suppressed terms for strong coupling \cite{Yamaguchi:2007ps}. 
 Paper IV provided a non-planar correction to this statement for the $\mathcal{N}=4$ case, 
 and helped to clarify a mismatch in an open problem concerning 1-loop corrections 
 in the holographic dual computation \cite{Faraggi:2014tna, Buchbinder:2014nia}. 
 Solving this problem is definitely a further non-trivial quantum rigorous test, 
 and we would gain a better understanding of divergences in stringy corrections. 
 There is a similar open problem concerning 1-loop matching for circular fundamental Wilson loop 
 in $\mathcal{N}=4$ SYM \cite{Kruczenski:2008zk, Kristjansen:2012nz, Bergamin:2015vxa, Forini:2015bgo, Faraggi:2016ekd, Forini:2017whz}.
%  \item 

\end{itemize}


Despite technical challenges, 
further studies on this $\mathcal{N}=2^*$SYM case are necessary for better rigorous understanding of the generic gauge/string duality.


